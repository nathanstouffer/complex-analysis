\documentclass[11pt]{exam}
\usepackage{amsmath,amssymb}
%\usepackage[margin=1.5in]{geometry}
\newcommand{\C}{\mathbb{C}}
\newcommand{\CC}{\mathbb{\hat{C}}}
\newcommand{\R}{\mathbb{R}}
\newcommand{\Z}{\mathbb{Z}}
\newcommand{\ds}{\displaystyle}
\DeclareMathOperator{\re}{Re}
\DeclareMathOperator{\im}{Im}
\DeclareMathOperator{\Log}{Log}
\DeclareMathOperator{\Arg}{Arg}
\begin{document}
\centerline{\Large M 472 -- Homework 2 -- Complex functions and differentiation}
\vspace{1ex}
\centerline{Nathan Stouffer}
\vspace{2ex}
\centerline{Due Monday, February 1, on Gradescope}
\vspace{3ex}
%\maketitle
\thispagestyle{empty}
\begin{questions}
\question Find a domain in the $z$-plane whose image under the
transformation $w=z^2$ is the square domain in the $w$-plane
bounded by the lines $u=1$, $u=2$, $v=1$, and $v=2$. (See
Section 14 in the textbook.) \\\\
We will identify the domain in the z-plane by finding the pre-image of the boundaries in the w plane.
First consider the line $w = 1 + iv = \sqrt{1 + v^2} \exp (i \arctan (v))$, taking the square root we get $z_1 = (1 + v^2)^{1/4} \exp (i (1/2) \arctan (v))$ for $v \in [1, 2]$.
Similarly for $w = 2 + i v$, we obtain $z_2 = (4 + v^2)^{1/4} \exp (i (1/2) \arctan (v/2))$ for $v \in [1,2]$.
For the horizontal lines, we begin with $w = u + i = \sqrt{u^2 + 1} \exp (i \arctan(1/x))$.
Taking a square root, we get $z_3 = (u^2 + 1)^{1/4} \exp (i (1/2 \arctan (1/x)))$ for $u \in [1,2]$.
Similarly for $w = u + i2$ we have $z_4 = (u^2 + 4)^{1/4} \exp (i (1/2 \arctan (2/x)))$ for $u \in [1,2]$.
So the preimage of the square in question is the region bounded by $z_1, z_2, z_3,$ and $z_4$.

\question Show that the function $\ds f(z) = \left( \frac{z}
{\bar{z}} \right)^2$ has the value $1$ at all nonzero points on
the real and imaginary axes, but that it has the value $-1$
at all nonzero points on the line $x=y$. Conclude that the
limit of $f(z)$ as $z$ tends to $0$ does not exist. \\\\
Consider $z = x \neq 0$.
Then we have $f(z) = \left( \frac{z}{\bar{z}} \right) ^2 = \left( \frac{x}{x} \right) ^2 = 1^2 = 1$.
Along the imaginary axis, we have $z = iy \neq 0 \implies f(z) = \left( \frac{iy}{-iy} \right) ^2 = \left( \frac{-y}{y} \right) ^2 = (-1)^2 = 1$.
Then we consider $z = x + ix \neq 0$ which gives $f(z) = \left( \frac{x + ix}{x - ix} \right) ^2 = \left( \frac{x + ix}{x - ix} * \frac{x+ix}{x+ix} \right) ^2 = \left( \frac{x^2 - x^2 + 2ix^2}{x^2 + x^2} \right) ^2 = i^2 = -1$.
But then the limit of $f(z)$ cannot exist as $z$ tends to 0 since there exist approaches to 0 that tend to different values.

\question Let $f$ be the function defined by
\[
f(z) =
\begin{cases}
  \bar{z}^2/z & \text{ when } z \ne 0, \\
  0 & \text{ when } z=0.
\end{cases}
\]
Show that if $z=0$, then $\Delta w / \Delta z = 1$ at each
nonzero point on the real and imaginary axes in the
$\Delta z$-plane. Then show that $\Delta w/ \Delta z = -1$ at
each nonzero point on the line $\Delta x = \Delta y$.
Conclude that $f'(0)$ does not exist. Note that to obtain this
result it is not sufficient to consider only horizontal and
vertical approaches to the origin in the $\Delta z$-plane. \\\\
For $z = 0$, we then have $\Delta w = f(0 + \Delta z) - f(0) = f(\Delta z)$.
We consider three cases for $\Delta z$.
First, we have $\Delta z = \Delta x \neq 0$.
Here $\Delta w / \Delta z = f(\Delta x) / \Delta x = \frac{\overline{\Delta x}^2}{\Delta x ^2} = \frac{\Delta x^2}{\Delta x ^2} = 1$.
Then consider $\Delta z = i \Delta y \neq 0$.
Then $\Delta w / \Delta z = \frac{\overline{i \Delta y}^2 / i \Delta y}{i \Delta y} = \frac{(-i \Delta y)^2}{(i \Delta y)^2} = \frac{-\Delta y^2}{- \Delta y ^2} = 1$.
In the final case, we have $\Delta x + i \Delta x \neq 0$.
Then $\Delta w / \Delta z = \frac{(\overline{\Delta x + i \Delta x})^2 / (\Delta x + i \Delta x)}{\Delta x + i \Delta x} = \frac{(\Delta x - i \Delta x)^2}{(\Delta x + i \Delta x)^2} = \frac{\Delta x ^2 - \Delta x ^2 - 2i\Delta x ^2}{\Delta x^2 - \Delta x^2 + 2i \Delta x ^2} = -1 $.
Since we have two approaches that tend to different limits, we know that $f'(0)$ does not exist.

\question Show from the definition of complex derivatives that
the functions $f(z) = \re z$ and $g(z) = \im z$ are not
complex differentiable at any point in the plane. \\\\
For the complex derivative to exist at a point, we must have the limit of the difference quotient converge as $\Delta z \to 0$.
We will show that this cannot hold for either of $f,g$ at any point $z \in \C$.
Let's begin with $f(z) = \re z$.
Take $\Delta z = \Delta x$, then $\lim _{\Delta x \to 0} \frac{f(z + \Delta x) - f(z)}{\Delta x} = \lim _{\Delta x \to 0} \Delta x / \Delta x = 1$.
But then for $\Delta z = i \Delta y$ we have $\lim _{\Delta y \to 0} \frac{f(z + i \Delta y) - f(z)}{i\Delta y} = \lim _{\Delta y \to 0} 0 / i \Delta y = 0$.
So the limit of $f$ cannot exist at any point $z \in C$. \\\\
Now consider $g(z) = \im z$.
Again take $\Delta z = \Delta x$ then $\lim _{\Delta x \to 0} \frac{g(z + \Delta x) - g(z)}{\Delta x} = \lim _{\Delta x \to 0} 0 / \Delta x = 0$.
And then for $\Delta z = i \Delta y$ we have $\lim _{\Delta y \to 0} \frac{g(z + i \Delta y) - g(z)}{i \Delta y} = \lim _{\Delta y \to 0} \frac{i \Delta y}{i \Delta y} = 1$.
So, again, the limit $g$ does not exist for any complex number $z$.

\question Using the definitions, rules for derivatives, or the
Cauchy-Riemann equations, determine where
the following functions are complex differentiable and find
$f'(z)$ where it exists.
\begin{parts}
  \part $f(z) = 1/(z^2+1)$
  \part $f(z) = x^2 + iy^2 $
  \part $f(z) = \sin x \cosh y - i \cos x \sinh y$
  \part $f(z) = \sin x \cosh y + i \cos x \sinh y$
\end{parts}

We consider the following functions.
\begin{parts}
  \part For $f(z) = 1/(z^2+1)$, the function $f$ is the quotient of two complex differentiable functions so it is complex differentiable for $z^2 + 1 \neq 0$ ($z \neq i$).
  The derivative is $f'(z) = -2z/(z^2 + 1)^2$.
  \part Take $f(z) = x^2 + iy^2 $, then $u = x^2$ and $v = y^2$.
  And we must have $u_x = 2x = v_y = 2y$ and $v_x = 0 = -u_y = 0$.
  This holds for $y = x$.
  Further, the partials are continuous so the derivative exists for $z$ such that $y = x$.
  The derivative is $f'(z) = 2x$.
  \part $f(z) = \sin x \cosh y - i \cos x \sinh y$ gives $u = \sin x \cosh y$ and $v = - \cos x \sinh y$.
  Then we have $u_x = \cos x \cosh y, u_y = \sin x \sinh y, v_x = \sin x \sinh y, v_y = - \cos x \cosh y$.
  We must have $\cos x \cosh y = - \cos x \cosh y \iff \cos x \cosh y = 0$ and $\sin x \sinh y = - \sin x \sinh y \iff \sin x \sinh y = 0$.
  But $\cos x$ and $\sin x$ are not 0 for the same $x$ so we must have $\cosh y = \sinh y = 0$.
  But this is also never the case for any $y$ so $f(z)$ is not complex differntiable anywhere.
  \part For $f(z) = \sin x \cosh y + i \cos x \sinh y$ we have almost the same equations as above.
  $u_x, u_y$ are the same but we have $v_x = - \sin x \sinh y$ and $v_y = - \cos x \cosh y$.
  These satisfy the Cauchy Riemann equations for all $x,y$ and are continuous partials, therefore $f'(z)$ exists for any complex number.
  It's formula is $f'(z) = \cos x \cosh y - i \sin x \sinh y$.
\end{parts}

\end{questions}
\end{document}
