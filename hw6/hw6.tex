\documentclass[11pt]{exam}
\usepackage{amsmath,amssymb}
%\usepackage[margin=1.5in]{geometry}
\newcommand{\C}{\mathbb{C}}
\newcommand{\CC}{\mathbb{\hat{C}}}
\newcommand{\R}{\mathbb{R}}
\newcommand{\Z}{\mathbb{Z}}
\newcommand{\ds}{\displaystyle}
\DeclareMathOperator{\re}{Re}
\DeclareMathOperator{\im}{Im}
\DeclareMathOperator{\Log}{Log}
\DeclareMathOperator{\Arg}{Arg}
\begin{document}
\centerline{\Large M 472 -- Homework 6}
\vspace{2ex}
\centerline{\Large Series}
\vspace{3ex}
\centerline{Due Monday, March 29, on Gradescope}
\vspace{3ex}
%\maketitle
\thispagestyle{empty}
\begin{questions}
  \question
  \begin{parts}
    \part Find the Maclaurin series (Taylor series centered at
    $z=0$) for the function $\ds f(z) = e^{-z^2}$. \\\\
    We know the Maclaurin series for $e^w = 1 + w + w^2 /2! + w^3 / 3! + \cdots$.
    Here we have $w = -z^2$ so the Maclaurin series turns into $1 - z^2 + z^4 /2! - z^6 /3! + z^8 /4! + \cdots$.
    \part Find the Maclaurin series for the function $\ds
    F(z) = \int_0^z e^{-\zeta^2} \, d\zeta$.\\ (Since the integrand is
    an entire function, the integral is independent of path, i.e.,
    $F(z)$ can be evaluated by integrating along any piecewise smooth
    path from $0$ to $z$.) \\\\
    Since the integral is independent of the path, we can just compute the antiderivative and evaluate $F(z) - F(0)$.
    The antiderivate can be computed term by term (as long as we have absolute convergence):
    \begin{align*}
        F(z) &= \int _0^z e^{-\zeta ^2} d \zeta \\
             &= \int _0^z \sum _{n=0}^\infty \frac{(-1)^n \zeta^{2n}}{n!} d\zeta \\
             &= \sum _{n=0}^\infty \frac{(-1)^n}{n!} \int _0^z \zeta ^{2n} d\zeta \\
             &= \left. \sum _{n=0}^\infty \frac{(-1)^n}{(2n+1)n!} \zeta ^{2n+1} \right| _0^z \\
             &= \sum _{n=0}^\infty \frac{(-1)^n}{(2n+1)n!} z ^{2n+1}
    \end{align*}
  \end{parts}

  \question
  Find the Laurent series of $\ds f(z) = \frac{z}{(z-1)(z+2)}$ in the
  annulus $\{ 1<|z|<2 \}$. (Hint: Your first step should be a partial
  fraction decomposition.) \\\\
  We have $\ds f(z) = z \frac{1}{(z-1)(z+2)} = z \left( \frac{1/3}{z-1} + \frac{-1/3}{z+2} \right) = \frac{z}{3} \left( \frac{1}{z-1} + \frac{1}{z+2} \right)$ by partial fraction decomposition.
  Then $\ds \frac{1}{z-1} = \frac{1}{z} \frac{1}{1-1/z} $ and recalling that $1 < |z|$ allows us to expand into a power series:
  $$\ds \frac{1}{z} \left( 1 + 1/z + 1/z^2 + 1/z^3 + \cdots \right) = 1/z \sum _{n=0}^\infty \frac{1}{z^n} = \sum _{n=0}^\infty \frac{1}{z^{n+1}} = \sum _{n=-\infty}^{-1} z^n $$
  On the other hand, $\ds \frac{1}{z+2} = \frac{1}{2} \frac{1}{1 - (-z/2)}$ and since $|z| < 2$ we expand to get
  $$ \frac{1}{2} \sum _{n=0}^\infty \left( \frac{-z}{2} \right)^n
  = \frac{1}{2} \sum _{n=0}^\infty \frac{(-1)^n z^n}{2^n}
  = \sum _{n=0}^\infty \frac{(-1)^n z^n}{2^{n+1}} $$
  This gives us the final Laurent series as
  $$ f(z)
  = z \left( \sum _{n=-\infty}^{-1} z^n + \sum _{n=0}^\infty \frac{(-1)^n z^n}{2^{n+1}} \right)
  = \sum _{n=-\infty}^{-1} z^{n+1} + \sum _{n=0}^\infty \frac{(-1)^n z^{n+1}}{2^{n+1}}
  $$

  \question
  \begin{parts}
    \part Find the Laurent series of $\ds f(z) = \frac{\sin z}{z}$,
    $\ds g(z) = \frac{\cos z}{z}$, and $\ds h(z) = \frac{\cos z}{z^2}$ in
    $\C^* = \C \setminus \{ 0 \}$. \\\\
    $$ f(z) = \frac{\sin z}{z} = \frac{z - z^3/3! + z^5/5! + \cdots}{z}
    = 1 - z^2/3! + z^4/5! - \cdots = \sum _{n=0}^\infty \frac{(-1)^n}{(2n+1)!} z^{2n} $$
    $$ g(z) = \frac{\cos z}{z} = \frac{1 - z^2/2! + z^4/4! - \cdots}{z}
    = 1/z - z/2! + z^3/4! - \cdots = \sum _{n=0}^\infty \frac{(-1)^n}{(2n)!} z^{2n-1} $$
    $$ h(z) = \frac{\cos z}{z^z} = \frac{1 - z^2/2! + z^4/4! - \cdots}{z^2}
    = 1/z^2 - 1/2! + z^2/4! - \cdots = \sum _{n=0}^\infty \frac{(-1)^n}{(2n)!} z^{2n-2}$$
    \part Which of the functions in (a) are entire? \\\\
    The only function from part (a) that is entire is $f(z)$.
    Both $g(z)$ and $h(z)$ have singularities at 0 so they are not entire.
    However, the power series expansion of $f(z)$ is entire, so $f(z)$ can be adapted to be entire.
    \part For each of the functions in (a), find the Laurent series of
    their antiderivative in $\C^*$ or explain why no such
    antiderivative exists. \\\\
    The antiderivative of $f(z)$ is $F(z) = \sum _{n=0}^\infty \frac{(-1)^n}{(2n+1)(2n+1)!} z^{2n+1}$ and the antiderivate of $h(z)$ is $H(z) = \sum _{n=0}^\infty \frac{(-1)^n}{(2n-1)(2n)!} z^{2n-1}$.
    The function $g(z)$ has no antiderivate since the integral of $1/z$ has a branch cut.
  \end{parts}

  \question Assume that $\ds f(z) = \sum_{n=-\infty}^\infty a_n z^n$
  is a Laurent series which converges in some annulus containing the
  circle $\{ |z|=r \}$. Find $\ds \int_{|z|=r} f(z) \, dz$ and $\ds
  \int_{|z|=r} \overline{f(z)} \, dz$ in terms of the coefficients
  $(a_n)$ and the radius $r$. \\ (Hint for the second integral: On the
  circle $|z|=r$ we have that $z \bar{z} = r^2$, so $\bar{z} =
  \frac{r^2}{z}$.) \\\\
  We wish to compute $\int _{|z|=r} f(z) dz$ given that $f(z)$ has a Laurent series.
  $$\int _{|z|=r} f(z) dz = \int _{|z|=r} \sum _{n=-\infty}^\infty a_n z^n dz = \sum _{n=-\infty}^\infty a_n \int _{|z| = r} z^n dz = 2 \pi i a_{-1}$$
  where the final step is justified since $\int _{|z| = r} z^n dz = 0$ when $n \neq -1$ and $\int _{|z| = r} ^n dz = 2 \pi i$ when $n = -1$.
  Now for $\overline{f(z)}$ we use the hint that $\bar z = r^2/z$:
  $$ \int _{|z|=r} \overline{f(z)} dz = \int _{|z|=r} \sum _{n=-\infty}^\infty \overline{a_n z^n} dz
  = \int _{|z|=r} \sum _{n=-\infty}^\infty \overline{a_n} (\overline{z})^n dz
  = \sum _{n=-\infty}^\infty \overline{a_n} \int _{|z|=r} \left( \frac{r^2}{z} \right) ^n dz$$
  $$
  = \sum _{n=-\infty}^\infty \overline{a_n} r^{2n} \int _{|z|=r} z^{-n} dz
  = 2 \pi i r^2 \overline{a_1} $$
  where the final simplification is justified similarly to the previous integral.

\end{questions}
\end{document}
