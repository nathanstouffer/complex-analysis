\documentclass[11pt]{exam}
\usepackage{amsmath,amssymb,multicol}
\usepackage[margin=.9in]{geometry}
\newcommand{\C}{\mathbb{C}}
\newcommand{\CC}{\mathbb{\hat{C}}}
\newcommand{\R}{\mathbb{R}}
\newcommand{\Z}{\mathbb{Z}}
\newcommand{\ds}{\displaystyle}
\DeclareMathOperator{\re}{Re}
\DeclareMathOperator{\im}{Im}
\DeclareMathOperator{\Log}{Log}
\DeclareMathOperator{\Arg}{Arg}
\DeclareMathOperator*{\Res}{Res}
\begin{document}
\centerline{\Large M 472 -- Homework 7}
\vspace{2ex}
\centerline{\Large Series}
\vspace{3ex}
\centerline{Due Friday, April 23, on Gradescope}
\vspace{3ex}
%\maketitle
\thispagestyle{empty}
\begin{questions}
  \question In each case, write the principal part of the function at
  its isolated singular point (i.e., the part of the Laurent series
  with negative exponents), and determine whether that point is a
  removable singularity, a pole, or an essential singularity.
  \begin{multicols}{2}
    \begin{parts}
      \part $\ds z^2 \sin \frac1z$
      \part $\ds \frac{z^2}{1+z}$
      \part $\ds \frac{1-\cos z}{z^2}$
      \part $\ds \frac{1-\cos z}{z^3}$
    \end{parts}
  \end{multicols}
  \question Evaluate the following residues.
  \begin{multicols}{2}
    \begin{parts}
      \part $\ds \Res_{z=2i} \left[ \frac{1}{z^2+4} \right]$
      \part $\ds \Res_{z=2i} \left[ \frac{e^{ix}}{(z^2+1)(z^2+4)}
      \right]$
      \part $\ds \Res_{z=0} \left[ z^2 \sin \frac1z \right]$
      \part $\ds \Res_{z=0} \left[ \frac{\sin z}{1-\cos z} \right]$
    \end{parts}
  \end{multicols}
  
  \question Use residues to find the following integrals:
  \begin{multicols}{2}
    \begin{parts}
      \part $\ds \int_{-\infty}^\infty \frac{dx}{x^2+4}$
      \part $\ds \int_{-\infty}^\infty \frac{dx}{(x^2+1)(x^2+4)}$
      \part $\ds \int_{-\infty}^\infty \frac{\cos x}{(x^2+1)(x^2+4)}
      \, dx$
      \part $\ds \int_0^{2\pi} \frac{\cos \theta}{5+4 \cos \theta} \,
      d\theta$
    \end{parts}
  \end{multicols}

  \question Consider the function $\ds f(z) = \frac{\csc (\pi z)}{z^2}
  = \frac{1}{z^2 \sin (\pi z)}$. Following the example in class used
  to evaluate the series $\ds \sum_{n=1}^\infty \frac{1}{n^2}$, we
  define $D_N$ to be the square region in the complex plane with
  vertices $\pm (N+1/2) \pm i (N+1/2)$. Again, just as in class, the
  $ML$-estimate, combined with explicit formulas for $f(z)$ on the
  boundary $\partial D_N$, can be used to show that $\ds
  \lim_{N\to\infty} \int_{\partial D_N} f(z) \, dz = 0$. (You can take
  this limit for granted, no need to show this.) Applying the Residue
  Theorem, as we did in the similar example in class, you will get the
  explicit value of some infinite series. Which infinite series is this,
  and what is its sum?
\end{questions}
\end{document}
