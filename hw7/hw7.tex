\documentclass[11pt]{exam}
\usepackage{amsmath,amssymb,multicol}
\usepackage[margin=.9in]{geometry}
\newcommand{\C}{\mathbb{C}}
\newcommand{\CC}{\mathbb{\hat{C}}}
\newcommand{\R}{\mathbb{R}}
\newcommand{\Z}{\mathbb{Z}}
\newcommand{\ds}{\displaystyle}
\DeclareMathOperator{\re}{Re}
\DeclareMathOperator{\im}{Im}
\DeclareMathOperator{\Log}{Log}
\DeclareMathOperator{\Arg}{Arg}
\DeclareMathOperator*{\Res}{Res}
\begin{document}
\centerline{\Large M 472 -- Homework 7}
\vspace{2ex}
\centerline{\Large Series}
\vspace{3ex}
\centerline{Due Friday, April 23, on Gradescope}
\vspace{3ex}
%\maketitle
\thispagestyle{empty}
\begin{questions}
  \question In each case, write the principal part of the function at
  its isolated singular point (i.e., the part of the Laurent series
  with negative exponents), and determine whether that point is a
  removable singularity, a pole, or an essential singularity.
  \begin{multicols}{2}
    \begin{parts}
      \part $\ds z^2 \sin \frac1z = z^2 * \left( 1/z - \frac{(1/z)^3}{3!}  + \frac{(1/z)^5}{5!} - \cdots \right)$ \\
      $\ds = z - \frac{1}{z*3!} + \frac{1}{z^3 * 5!} - \cdots $ \\
      So we have an essential singularity at $z = 0$.
      \part $\ds \frac{z^2}{1+z} = \frac{z^2}{z*(1+1/z)} = \frac{z}{1- (-1/z)}$ \\
      $\ds = z * \left( 1 - 1/z + 1/z^2 - 1/z^3 + \cdots  \right)$ \\
      $\ds = z - 1 + 1/z - 1/z^2 + \cdots $. \\
      So we have another essential singularity at $z = 0$.
      \part $\ds \frac{1-\cos z}{z^2} = \frac{1 - (1 - z^2 /2 + z^4 / 4! - \cdots )}{z^2}$ \\
      $ \ds = \frac{z^2 / 2 - z^4 / 4! + z^6 / 6! - \cdots}{z^2}$ \\
      $ \ds = 1/2 - z^2 / 4! + z^4 / 6! - \cdots $ \\
      So we have a removable singularity here.
      \part $\ds \frac{1-\cos z}{z^3} = \frac{1 - (1 - z^2 /2 + z^4 / 4! - \cdots )}{z^3}$ \\
       $ \ds = \frac{z^2 / 2 - z^4 / 4! + z^6 / 6! - \cdots}{z^3}$ \\
      $ \ds = 1/2z - z/ 4! + z^3 / 6! - \cdots$ \\
      So we have pole at $z = 0$.
    \end{parts}
  \end{multicols}
	
  \question Evaluate the following residues.
  \begin{multicols}{2}
    \begin{parts}
      \part $\ds \Res_{z=2i} \left[ \frac{1}{z^2+4} \right]$
      \part $\ds \Res_{z=2i} \left[ \frac{e^{iz}}{(z^2+1)(z^2+4)}
      \right]$
      \part $\ds \Res_{z=0} \left[ z^2 \sin \frac1z \right]$
      \part $\ds \Res_{z=0} \left[ \frac{\sin z}{1-\cos z} \right]$
    \end{parts}
  \end{multicols}

	Here are my answers.
	\begin{parts}
		\part $\ds \Res_{z=2i} \left[ \frac{1}{z^2+4} \right] = \Res _{z = 2i} \frac{1}{(z+2i)(z-2i)}
		= \lim _{z \to 2i} \frac{1}{z+2i} = -i/4$.
		\part $\ds \Res_{z=2i} \left[ \frac{e^{iz}}{(z^2+1)(z^2+4)} \right]
		= \Res_{z = 2i} \frac{e^{iz}}{(z^2 + 1)(z+2i)(z-2i)} = \lim_{z \to 2i} \frac{e^{iz}}{(z^2 + 1)(z+2i)} = i/12e^2 $
		\part $\ds \Res_{z=0} \left[ z^2 \sin \frac1z \right] = -1/3! = -1/6$ since from the Laurent series in the previous problem.
		\part $\ds \Res_{z=0} \left[ \frac{\sin z}{1-\cos z} \right]
		= \Res_{z=0} \frac{z - z^3/3! + z^5/5! - \cdots}{1-(1-z^2/2! + z^4/4! - \cdots)}
		= \Res_{z=0} \frac{z - z^3/3! + z^5/5! - \cdots}{-z^2/2! + z^4/4! - \cdots)}$ \\
		$\ds = \lim_{z \to 0} \frac{1 - z^2/3! + z^4/5! - \cdots}{1/2! - z^2 / 4! + z^4 / 6! - \cdots}
		= 2$ 
	\end{parts}
  
  \question Use residues to find the following integrals:
  \begin{multicols}{2}
    \begin{parts}
      \part $\ds \int_{-\infty}^\infty \frac{dx}{x^2+4}$
      \part $\ds \int_{-\infty}^\infty \frac{dx}{(x^2+1)(x^2+4)}$
      \part $\ds \int_{-\infty}^\infty \frac{\cos x}{(x^2+1)(x^2+4)}
      \, dx$
      \part $\ds \int_0^{2\pi} \frac{\cos \theta}{5+4 \cos \theta} \,
      d\theta$
    \end{parts}
  \end{multicols}

	\begin{parts}
		\part For this one consider the complex integral over the boundary of $D_R$, the disk of radius $R$ only in the upper half-plane.
		$\ds \int_{\partial D_R} \frac{1}{z^2 + 4} \, dz $.
		By the residue theorem, this integral equals $2 \pi i * \Res_{z = 2i} \frac{1}{z^2 + 4} = 2 \pi i \frac{-i}{4} = \pi / 2$.
		Now we show that integral over the arc is 0 (meaning that $\int_{\infty}^\infty \frac{1}{x^2 + 4} \, dx = \pi / 2)$.
		For the remainder of problem 3, let $C_1$ be the arc of $D_R$ parameterized by $z(t) = R e^{it}$ for $0 \leq t \leq \pi$.
		Then, by the ML-estimate
		$\left| \int _{C_1} \frac{1}{z^2+4} \, dz \right| 
		\leq \int _{C_1} \left| \frac{1}{z^2+4} \right| \, dz 
		\leq \int _{C_1} \left| \frac{1}{z^2} \right| \, dz 
		= \int _{C_1} \frac{1}{R^2} \, dz = pi R / R^2 = \pi/ R $ which goes to 0 as $R \to \infty$.
		\part Let $\ds f(z) = \frac{1}{(z^2+1)(z^2+4)}$.
		Then
		$ \ds \int_{\partial D_R} \frac{1}{(z^2+1)(z^2+4)} \, dz = 2 \pi i \left[ \Res_{z = 2i} f(z) + \Res_{z=i} f(z) \right] = 2 \pi i \left[ 1/(-3*4i) + 1/(2i*3) \right] = \pi / 6 $.
		Then we have the ML estimate 
		$\left| \int _{C_1} f(z) \, dz \right| 
		\leq \int _{C_1} \left| f(z) \right| \, dz 
		\leq \int_{C_1} 1/z^4 \, dz = \pi R / R^4 = \pi / R^3$ which goes to 0 as $R$ goes to $\infty$.
		\part Let $\ds g(z) = \frac{e^{iz}}{(z^2+1)(z^2+4)}$.
		Then $\ds \int_{\partial D_R} \frac{e^{iz}}{(z^2+1)(z^2+4)} \, dz= 2 \pi i \left[ \Res_{z = 2i} g(z) + \Res_{z=i} g(z) \right] = \frac{\pi}{6e^2} (2e - 1) $.
		This is entirely real so there is no need to take the real part of this to get our actual answer.
		Also, we need the ML estimate:
		$\ds \left| \int _{C_1} g(z) \, dz \right| 
		\leq \int _{C_1} \left| g(z) \right| \, dz 
		\leq \int_{C_1} 1/z^4 \, dz = \pi R / R^4 = \pi / R^3$
		which goes to 0 and $R \to \infty$.
		\part I struggled on this one and I am already a day late so I didn't type it up.
		Sorry!
	\end{parts} 

  \question Consider the function $\ds f(z) = \frac{\csc (\pi z)}{z^2}
  = \frac{1}{z^2 \sin (\pi z)}$. Following the example in class used
  to evaluate the series $\ds \sum_{n=1}^\infty \frac{1}{n^2}$, we
  define $D_N$ to be the square region in the complex plane with
  vertices $\pm (N+1/2) \pm i (N+1/2)$. Again, just as in class, the
  $ML$-estimate, combined with explicit formulas for $f(z)$ on the
  boundary $\partial D_N$, can be used to show that $\ds
  \lim_{N\to\infty} \int_{\partial D_N} f(z) \, dz = 0$. (You can take
  this limit for granted, no need to show this.) Applying the Residue
  Theorem, as we did in the similar example in class, you will get the
  explicit value of some infinite series. Which infinite series is this,
  and what is its sum? \\\\
  First note that $f(z)$ has singularities at $z = 0, \pm 1, \pm 2, \cdots, \pm N $ in the domain $D_N$.
  From mathematica, we know that $\Res_{z=0} f(z) = \pi / 6$.
  For some $k \in \Z \setminus \{ 0 \}$ we have
  $\ds \Res_{z=k} \frac{ \csc (\pi z)}{z^2} = \Res_{z=k} \frac{1/z^2}{\sin (\pi z)} = \frac{1/k^2}{\pi \cos (\pi k)} = \frac{(-1)^k}{\pi k^2}$.
  By the residue theorem, we have $ \ds \frac{1}{2 \pi i} \int_{\partial D_N} f(z) \, dz = \pi / 6 + \sum_{k=1}^N \left( \frac{(-1)^k}{\pi k^2} + \frac{(-1)^{-k}}{\pi k^2} \right) = \pi /6 + \sum_{k=1}^N \frac{2(-1)^k}{\pi k^2} $.
  Then taking $N \to \infty$ gives $ \ds 0 = \pi / 6 + \sum_{k=1}^\infty \frac{2(-1)^k}{\pi k^2}$ which means that $\ds \pi ^2 / 12 = - \sum_{k=1}^\infty \frac{(-1)^k}{k^2} = \sum_{k=1}^\infty \frac{(-1)^{k+1}}{k^2} = 1 - 1/4 + 1/9 - 1/16 + \cdots $.
  This is the alternating series version of the Basel problem.
\end{questions}
\end{document}
